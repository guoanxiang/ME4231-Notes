\section{Week 1}
\subsection{Deriving NS equations}

\begin{equation} %\label{NS equations in compact form}
    \frac{\partial \mathbf{u}}{\partial t} + (\mathbf{u} \cdot \nabla)\mathbf{u} = -\frac{1}{\rho} \nabla p + \nu \nabla ^{2} \mathbf{u} + \mathbf{f}
\end{equation}


On the left-hand side, there consists of the time derivative term. 

\begin{equation}
    \frac{\partial \mathbf{u}}{\partial t} =\begin{pmatrix}\frac{\partial u}{\partial t} \\[6pt] \frac{\partial v}{\partial t} \\[6pt] \frac{\partial w}{\partial t}\end{pmatrix}
\end{equation}

There also consists of the convective term. 

\begin{equation}
    (\mathbf{u} \cdot \nabla)\mathbf{u} = \left( \begin{pmatrix}u \\v \\ w\end{pmatrix}\cdot\begin{pmatrix}\frac{\partial}{\partial x} \\[6pt] \frac{\partial}{\partial y} \\[6pt] \frac{\partial}{\partial z}\end{pmatrix}\right)\begin{pmatrix}u \\v \\ w\end{pmatrix}=\left( u \frac{\partial}{\partial x} + v \frac{\partial}{\partial y} + w \frac{\partial}{\partial z} \right)\begin{pmatrix}u \\v \\ w\end{pmatrix} =\begin{pmatrix}\left( u \frac{\partial}{\partial x} + v \frac{\partial}{\partial y} + w \frac{\partial}{\partial z} \right)u \\[6pt] \left( u \frac{\partial}{\partial x} + v \frac{\partial}{\partial y} + w \frac{\partial}{\partial z} \right)v \\[6pt] \left( u \frac{\partial}{\partial x} + v \frac{\partial}{\partial y} + w \frac{\partial}{\partial z} \right)w\end{pmatrix}
\end{equation}

\begin{equation}
    (\mathbf{u} \cdot \nabla)\mathbf{u} = \begin{pmatrix} u \frac{\partial u}{\partial x} + v \frac{\partial u}{\partial y} + w \frac{\partial u}{\partial z} \\[6pt]  u \frac{\partial v}{\partial x} + v \frac{\partial v}{\partial y} + w \frac{\partial v}{\partial z}  \\[6pt]  u \frac{\partial w}{\partial x} + v \frac{\partial w}{\partial y} + w \frac{\partial w}{\partial z} \end{pmatrix}
\end{equation}

On the right-hand side, there consists of the pressure gradient term.  

\begin{equation}
-\frac{1}{\rho} \nabla p=-\frac{1}{\rho}\begin{pmatrix}
    \frac{\partial p}{\partial x} \\[6pt]
    \frac{\partial p}{\partial y} \\[6pt]
    \frac{\partial p}{\partial z}
\end{pmatrix}=
\begin{pmatrix}
    -\frac{1}{\rho}\frac{\partial p}{\partial x} \\[6pt]
    -\frac{1}{\rho}\frac{\partial p}{\partial y} \\[6pt]
    -\frac{1}{\rho}\frac{\partial p}{\partial z}
\end{pmatrix}
\end{equation}

Also, the viscous friction term on RHS. 

\begin{equation}
\nu \nabla ^{2} \mathbf{u} 
= \nu\left(\frac{\partial^{2}}{\partial x^{2}}+\frac{\partial^{2}}{\partial y^{2}}+\frac{\partial^{2}}{\partial z^{2}}\right)\begin{pmatrix}
    u \\
    v \\
    w
\end{pmatrix}
=\nu\begin{pmatrix}
    \frac{\partial^{2}u}{\partial x^{2}}+\frac{\partial^{2}u}{\partial y^{2}}+\frac{\partial^{2}u}{\partial z^{2}} \\[6pt]
    \frac{\partial^{2}v}{\partial x^{2}}+\frac{\partial^{2}v}{\partial y^{2}}+\frac{\partial^{2}v}{\partial z^{2}} \\[6pt]
    \frac{\partial^{2}w}{\partial x^{2}}+\frac{\partial^{2}w}{\partial y^{2}}+\frac{\partial^{2}w}{\partial z^{2}}
\end{pmatrix}
\end{equation}

\begin{figure}[h]
    \centering
    \includegraphics[width=0.7\linewidth]{references/screenshot001}
    \caption{The full form of Naiver Stokes Momentum Equation}
    \label{fig:full form NS equations}
\end{figure}


This gives the equations listed in figure \ref{fig:full form NS equations}. Conversion of matrices above to PDE is not listed. 

\subsubsection{Linear Momentum Equations}
\begin{equation}
    \sum \mathbf{F} = \sum (\dot{m}\cdot \mathbf{v})_{\mathrm{out}} - \sum (\dot{m}\cdot \mathbf{v})_{\mathrm{in}}
\end{equation}

Assumptions: steady flow. 

Net force acting on a system = \textbf{rate of change} of linear momentum. 

Can separate into 2 components, x and y, if needed. 

\begin{equation}
    \dot{m} = \rho A v = m v
\end{equation}

Using equation above, and adding $m$ we obtain 

Relating linear momentum $p = mv$ to the NS equations is shown below. Multiply density to all terms 

For first term in NS equations, 

\begin{equation}
    \rho \frac{\partial u}{\partial t} = \frac{m}{V} \frac{\partial u}{\partial t}
\end{equation}

We obtain forces on a volume. 

For second term, 

\begin{equation}
    \rho u \cdot \frac{\partial u}{\partial x} = 
\end{equation}

\begin{center}
    \colorbox{pink}{continue with derivation of NS equations, slide 27}\\
    \colorbox{pink}{continue with non-dimensionalized equations}
\end{center}

\subsection{Mathematics}
Prerequisite: Gradient operator $\nabla$, $\nabla k$, where $k$ is a constant, divergence $\nabla \cdot \mathbf{V}$, Laplacian $\nabla^{2} = \Delta = \nabla \cdot \nabla$ (Note this is a scalar, treat this like a mathematical operator, similar to $\frac{\partial^{2}}{\partial x^{2}}$), curl $\nabla \times \mathbf{V}$. 

\subsubsection{Vorticity}
Vorticity is a vector, $\boldsymbol{\zeta}$. (In notes expressed as $\boldsymbol{\omega}$ but do not wish to confuse with angular velocity). 

\begin{equation}
    \boldsymbol{\zeta} = \nabla \times \mathbf{v} = 
    \begin{pmatrix}
        \frac{\partial}{\partial x}\\[6pt]
        \frac{\partial}{\partial y}\\[6pt]
        \frac{\partial}{\partial z}\\
    \end{pmatrix}
    \times
    \begin{pmatrix}
        u\\
        v\\
        w\\
    \end{pmatrix}
\end{equation}

Where $\mathbf{v}$ is a velocity vector. 

\textbf{Recap on Angular Kinematics}

\begin{equation}
    s = r \theta
\end{equation}
\begin{equation}
    v = r \omega
\end{equation}
\begin{equation}
    a = r \alpha
\end{equation}


\textbf{Relate Angular Velocity to Vorticity}

Below is a prerequisite on triple product rule of vectors. 
\begin{equation}
    \mathbf{a} \times \left(\mathbf{b} \times \mathbf{c}\right) = -\mathbf{c}\times\left(\mathbf{a}\times\mathbf{b}\right) = 
\left(\mathbf{c}\cdot\mathbf{a}\right) \mathbf{b} - \left(\mathbf{c}\cdot\mathbf{b}\right) \mathbf{a}
\end{equation}

Utilizing the above rule, 
\begin{equation}
    \mathbf{v} = \boldsymbol{\omega} \times \mathbf{r}
\end{equation}

\begin{align}
    \nabla \times (\boldsymbol{\omega} \times \mathbf{r})
    &= (\nabla \cdot \mathbf{r})\boldsymbol{\omega}-(\nabla \cdot \boldsymbol{\omega}) \mathbf{r} \\
    &=\boldsymbol{\omega}\left(\frac{\partial x}{\partial x} + 
    \frac{\partial y}{\partial y} + \frac{\partial z}{\partial z}\right) - \left(\omega_{x}\frac{\partial}{\partial x} + \omega_{y}\frac{\partial}{\partial y} + 
    \omega_{z}\frac{\partial}{\partial z}\right)
    \begin{pmatrix}
        x \\
        y \\
        z 
    \end{pmatrix}
\end{align} 

Due to $x$, $y$, and $z$ being independent of each other, $\frac{\partial x}{\partial y} \, , \, \frac{\partial x}{\partial z} \, , \, \frac{\partial y}{\partial x} \, , \, \frac{\partial y}{\partial z} \, , \, \frac{\partial z}{\partial x} \; \mathrm{and} \; \frac{\partial z}{\partial y} = 0$ 
\begin{align}
    \nabla \times (\boldsymbol{\omega} \times \mathbf{r}) 
    &=3\boldsymbol{\omega} - \boldsymbol{\omega} \\
    \boldsymbol{\zeta}=\nabla\times\mathbf{v}&=2\boldsymbol{\omega}
\end{align}

If flow is irrotational, i.e. no vorticity, 
\begin{equation}
    \boldsymbol{\zeta}=\boldsymbol{\omega} = \mathbf{0}
\end{equation}

\subsubsection{Potential Flow}
\begin{align}
    \nabla \times \nabla k &= 
    \begin{pmatrix}
        \frac{\partial}{\partial x}\\[6pt]
        \frac{\partial}{\partial y}\\[6pt]
        \frac{\partial}{\partial z}
    \end{pmatrix}
    \times
    \begin{pmatrix}
        \frac{\partial k}{\partial x}\\[6pt]
        \frac{\partial k}{\partial y}\\[6pt]
        \frac{\partial k}{\partial z}
    \end{pmatrix}
\end{align}
where k is a scalar. After cross product is attempted, it is clear that each entry in vector gives 0. 
\begin{align}
    \nabla \times \nabla k = \mathbf{0}
\end{align}
It is easy to see that if $\nabla \times \mathbf{v} = \textbf{0}$, then $\mathbf{v} = \nabla k$, gradient of a scalar. 

\subsubsection{Kinematic Viscosity and Dynamic Viscosity}
Kinematic viscosity $\nu$ and dynamic viscosity $\mu$ is related by the formula below. 
\begin{equation}
    \nu = \frac{\mu}{\rho}
\end{equation}
Since density changes with temperature, both $\nu$ and $\mu$ is temperature dependent. 

\subsubsection{Shear Stress in a Fluid}
Shear stress $\tau_{xy}$ is given by the equation below. 
\begin{equation}
    \tau_{xy} = \mu \cdot \left(\frac{\partial u}{\partial y}\right)
\end{equation}

Fluid flows in the $x$ direction, from left to right. Fluid gains in elevation in $y$ direction, moves away from boundary plate. 

If flow is inviscid, $\mu = 0$, then shear stress $\tau = 0$. 

More on this concept in ME2134. 

\subsubsection{Skin Friction Drag}
Skin friction drag $D$ is given by the equation below. 
\begin{equation}
    D =\int\tau \cdot dA
\end{equation}

\subsubsection{Formulas}
\begin{center}
    \begin{tabular}{|c | c | c | c|}
        \hline
        Reynolds Number & Mach Number & Speed of Sound & Knudsen Number \\
        \hline
        $Re = \frac{\rho \cdot u \cdot L}{\mu} = \frac{u \cdot L}{\nu}$ & $M = \frac{V}{a}$ & $a=\sqrt{\gamma R T}$ & $Kn = \frac{\textrm{Mean \, Free \, Path}}{\textrm{Characteristic \, Length \, Scale}} = \frac{\lambda}{d}$\\
        \hline
    \end{tabular}
\end{center}
